\usepackage[
    a4paper, 
    mag=1000, 
    left=2.5cm, 
    right=1cm, 
    top=2cm, 
    bottom=2cm, 
    headsep=0.7cm, 
    footskip=1cm
]{geometry}

\usepackage[utf8]{inputenc}
\usepackage[english,russian]{babel}
\usepackage{indentfirst}
\usepackage[dvipsnames]{xcolor}
\usepackage[colorlinks]{hyperref}
\usepackage{listings} 
\usepackage{caption}

\usepackage{graphicx}
\usepackage{float}
\usepackage{wrapfig}
\graphicspath{{images/}}

\usepackage{mathtools, nccmath}

\lstset{ % Настройки вида листинга
	inputencoding=utf8, 
	extendedchars=\true, 		% поддержка кириллицы 
	keepspaces = true, 			% поддержка пробелов 
	language=mathematica,       % выбор языка для подсветки 
	basicstyle=\small\sffamily, % размер и начертание шрифта для подсветки кода
	numbers=left,               % где поставить нумерацию строк (слева\справа)
	numberstyle=\small,         % размер шрифта для номеров строк
	stepnumber=1,               % размер шага между двумя номерами строк
	numbersep=7pt,              % как далеко отстоят номера строк от подсвечиваемого кода
	showspaces=false,           % показывать или нет пробелы специальными отступами
	showstringspaces=false,     % показывать или нет пробелы в строках
	showtabs=false,             % показывать или нет табуляцию в строках
	frame=false,               	% рисовать рамку вокруг кода
	tabsize=2,                  % размер табуляции по умолчанию равен 2 пробелам
	captionpos=t,               % позиция заголовка вверху [t] или внизу [b] 
	breaklines=true,            % автоматически переносить строки (да\нет)
	breakatwhitespace=false,    % переносить строки только если есть пробел
}

\sloppy             % Избавляемся от переполнений
\hyphenpenalty=1000 % Частота переносов
\clubpenalty=10000  % Запрещаем разрыв страницы после первой строки абзаца
\widowpenalty=10000 % Запрещаем разрыв страницы после последней строки абзаца

