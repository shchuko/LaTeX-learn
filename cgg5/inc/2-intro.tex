1. Познакомится с инструментами создания векторных примитивов;\\


2. Научится использовать инструменты трансформации векторных объектов;\\


3. Научиться применять теоретико-множественны операции над векторными примитивами
для построения сложных объектов;\\


4. Построить фигуры, приведенные в работе, используя векторные примитивы, 
изменение параметров заливки и обводки, операции трансформации, операции
выравнивания и теоретико-множественные операции над векторными примитивами;\\


5. Для каждой фигуры описать порядок ее построения;\\


6. Прикрепить к отчету файлы в формате Inkscape с построенными фигурами.
