\textbf{Выполнены поставленные задачи:}
\begin{itemize}
    \item исследовано распределение Максвелла по модулю скорости при различных температурах для различных газов; 
    \item исследовано распределение Максвелла по проекции (составляющей) скорости при различных температурах для различных газов; 
    \item для обеих распределений подтверждено визуализацией, 
        что число молекул с большими скоростями прямо пропорционально температуре 
        и обратно пропорционально массе молекулы;
    \item проверена и подверждена нормировка распределений;
    \item оценен процент молекул, имеющих скорости $[0, v_{\text{ср.кв.}}]$;
    \item определена наиболее вероятная скорость молекул при различных температурах;
    \item определен процент молекул, имеющих скорость, отличающуюся от наиболее вероятной на 1\%, 5\%.
    \item исследовано распределение Больцмана для различных газов при постоянной температуре;
    \item исследовано распрелеление Больцмана для различных газов при убывающей с высотой температуре;
    \item для обоих случаев подтверждено визуалицаей, что с увеличением высоты концентрация убывает,
        скорость убывания прямо пропорциональна массе молекулы.
\end{itemize}

\textbf{Для выполнения были использованы инструменты Wolftam Mathematica:}
\begin{itemize}
    \item Plot - построение двумерных графиков функций;
    \item Piecewise - кусочное заданий функций;
    \item Integrate - интегрирование функций.
\end{itemize}


